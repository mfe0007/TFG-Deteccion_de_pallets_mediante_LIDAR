\documentclass[a4paper,12pt,twoside]{memoir}

% Castellano
\usepackage[spanish,es-tabla]{babel}
\selectlanguage{spanish}
\usepackage[utf8]{inputenc}
\usepackage[T1]{fontenc}
\usepackage{lmodern} % Scalable font
\usepackage{microtype}
\usepackage{placeins}

\RequirePackage{booktabs}
\RequirePackage[table]{xcolor}
\RequirePackage{xtab}
\RequirePackage{multirow}

% Links
\usepackage[colorlinks]{hyperref}
\hypersetup{
	allcolors = {red}
}

% Ecuaciones
\usepackage{amsmath}

% Rutas de fichero / paquete
\newcommand{\ruta}[1]{{\sffamily #1}}

% Párrafos
\nonzeroparskip


% Imagenes
\usepackage{graphicx}
\newcommand{\imagen}[2]{
	\begin{figure}[!h]
		\centering
		\includegraphics[width=0.9\textwidth]{#1}
		\caption{#2}\label{fig:#1}
	\end{figure}
	\FloatBarrier
}

\newcommand{\imagenflotante}[2]{
	\begin{figure}%[!h]
		\centering
		\includegraphics[width=0.9\textwidth]{#1}
		\caption{#2}\label{fig:#1}
	\end{figure}
}



% El comando \figura nos permite insertar figuras comodamente, y utilizando
% siempre el mismo formato. Los parametros son:
% 1 -> Porcentaje del ancho de página que ocupará la figura (de 0 a 1)
% 2 --> Fichero de la imagen
% 3 --> Texto a pie de imagen
% 4 --> Etiqueta (label) para referencias
% 5 --> Opciones que queramos pasarle al \includegraphics
% 6 --> Opciones de posicionamiento a pasarle a \begin{figure}
\newcommand{\figuraConPosicion}[6]{%
  \setlength{\anchoFloat}{#1\textwidth}%
  \addtolength{\anchoFloat}{-4\fboxsep}%
  \setlength{\anchoFigura}{\anchoFloat}%
  \begin{figure}[#6]
    \begin{center}%
      \Ovalbox{%
        \begin{minipage}{\anchoFloat}%
          \begin{center}%
            \includegraphics[width=\anchoFigura,#5]{#2}%
            \caption{#3}%
            \label{#4}%
          \end{center}%
        \end{minipage}
      }%
    \end{center}%
  \end{figure}%
}

%
% Comando para incluir imágenes en formato apaisado (sin marco).
\newcommand{\figuraApaisadaSinMarco}[5]{%
  \begin{figure}%
    \begin{center}%
    \includegraphics[angle=90,height=#1\textheight,#5]{#2}%
    \caption{#3}%
    \label{#4}%
    \end{center}%
  \end{figure}%
}
% Para las tablas
\newcommand{\otoprule}{\midrule [\heavyrulewidth]}
%
% Nuevo comando para tablas pequeñas (menos de una página).
\newcommand{\tablaSmall}[5]{%
 \begin{table}
  \begin{center}
   \rowcolors {2}{gray!35}{}
   \begin{tabular}{#2}
    \toprule
    #4
    \otoprule
    #5
    \bottomrule
   \end{tabular}
   \caption{#1}
   \label{tabla:#3}
  \end{center}
 \end{table}
}

%
% Nuevo comando para tablas pequeñas (menos de una página).
\newcommand{\tablaSmallSinColores}[5]{%
 \begin{table}[H]
  \begin{center}
   \begin{tabular}{#2}
    \toprule
    #4
    \otoprule
    #5
    \bottomrule
   \end{tabular}
   \caption{#1}
   \label{tabla:#3}
  \end{center}
 \end{table}
}

\newcommand{\tablaApaisadaSmall}[5]{%
\begin{landscape}
  \begin{table}
   \begin{center}
    \rowcolors {2}{gray!35}{}
    \begin{tabular}{#2}
     \toprule
     #4
     \otoprule
     #5
     \bottomrule
    \end{tabular}
    \caption{#1}
    \label{tabla:#3}
   \end{center}
  \end{table}
\end{landscape}
}

%
% Nuevo comando para tablas grandes con cabecera y filas alternas coloreadas en gris.
\newcommand{\tabla}[6]{%
  \begin{center}
    \tablefirsthead{
      \toprule
      #5
      \otoprule
    }
    \tablehead{
      \multicolumn{#3}{l}{\small\sl continúa desde la página anterior}\\
      \toprule
      #5
      \otoprule
    }
    \tabletail{
      \hline
      \multicolumn{#3}{r}{\small\sl continúa en la página siguiente}\\
    }
    \tablelasttail{
      \hline
    }
    \bottomcaption{#1}
    \rowcolors {2}{gray!35}{}
    \begin{xtabular}{#2}
      #6
      \bottomrule
    \end{xtabular}
    \label{tabla:#4}
  \end{center}
}

%
% Nuevo comando para tablas grandes con cabecera.
\newcommand{\tablaSinColores}[6]{%
  \begin{center}
    \tablefirsthead{
      \toprule
      #5
      \otoprule
    }
    \tablehead{
      \multicolumn{#3}{l}{\small\sl continúa desde la página anterior}\\
      \toprule
      #5
      \otoprule
    }
    \tabletail{
      \hline
      \multicolumn{#3}{r}{\small\sl continúa en la página siguiente}\\
    }
    \tablelasttail{
      \hline
    }
    \bottomcaption{#1}
    \begin{xtabular}{#2}
      #6
      \bottomrule
    \end{xtabular}
    \label{tabla:#4}
  \end{center}
}

%
% Nuevo comando para tablas grandes sin cabecera.
\newcommand{\tablaSinCabecera}[5]{%
  \begin{center}
    \tablefirsthead{
      \toprule
    }
    \tablehead{
      \multicolumn{#3}{l}{\small\sl continúa desde la página anterior}\\
      \hline
    }
    \tabletail{
      \hline
      \multicolumn{#3}{r}{\small\sl continúa en la página siguiente}\\
    }
    \tablelasttail{
      \hline
    }
    \bottomcaption{#1}
  \begin{xtabular}{#2}
    #5
   \bottomrule
  \end{xtabular}
  \label{tabla:#4}
  \end{center}
}



\definecolor{cgoLight}{HTML}{EEEEEE}
\definecolor{cgoExtralight}{HTML}{FFFFFF}

%
% Nuevo comando para tablas grandes sin cabecera.
\newcommand{\tablaSinCabeceraConBandas}[5]{%
  \begin{center}
    \tablefirsthead{
      \toprule
    }
    \tablehead{
      \multicolumn{#3}{l}{\small\sl continúa desde la página anterior}\\
      \hline
    }
    \tabletail{
      \hline
      \multicolumn{#3}{r}{\small\sl continúa en la página siguiente}\\
    }
    \tablelasttail{
      \hline
    }
    \bottomcaption{#1}
    \rowcolors[]{1}{cgoExtralight}{cgoLight}

  \begin{xtabular}{#2}
    #5
   \bottomrule
  \end{xtabular}
  \label{tabla:#4}
  \end{center}
}


















\graphicspath{ {./img/} }

% Capítulos
\chapterstyle{bianchi}
\newcommand{\capitulo}[2]{
	\setcounter{chapter}{#1}
	\setcounter{section}{0}
	\chapter*{#2}
	\addcontentsline{toc}{chapter}{#2}
	\markboth{#2}{#2}
}

% Apéndices
\renewcommand{\appendixname}{Apéndice}
\renewcommand*\cftappendixname{\appendixname}

\newcommand{\apendice}[1]{
	%\renewcommand{\thechapter}{A}
	\chapter{#1}
}

\renewcommand*\cftappendixname{\appendixname\ }

% Formato de portada
\makeatletter
\usepackage{xcolor}
\newcommand{\tutor}[1]{\def\@tutor{#1}}
\newcommand{\course}[1]{\def\@course{#1}}
\definecolor{cpardoBox}{HTML}{E6E6FF}
\def\maketitle{
  \null
  \thispagestyle{empty}
  % Cabecera ----------------
\noindent\includegraphics[width=\textwidth]{cabecera}\vspace{1cm}%
  \vfill
  % Título proyecto y escudo informática ----------------
  \colorbox{cpardoBox}{%
    \begin{minipage}{.8\textwidth}
      \vspace{.5cm}\Large
      \begin{center}
      \textbf{TFG del Grado en Ingeniería Informática}\vspace{.6cm}\\
      \textbf{\LARGE\@title{}}
      \end{center}
      \vspace{.2cm}
    \end{minipage}

  }%
  \hfill\begin{minipage}{.20\textwidth}
    \includegraphics[width=\textwidth]{escudoInfor}
  \end{minipage}
  \vfill
  % Datos de alumno, curso y tutores ------------------
  \begin{center}%
  {%
    \noindent\LARGE
    Presentado por \@author{}\\ 
    en Universidad de Burgos --- \@date{}\\
    Tutor: \@tutor{}\\
  }%
  \end{center}%
  \null
  \cleardoublepage
  }
\makeatother

\newcommand{\nombre}{Mario Flores Espiga} %%% cambio de comando

% Datos de portada
\title{Detección de palets mediante LIDAR}
\author{\nombre}
\tutor{Jesus Enrique Sierra García}
\date{\today}

\begin{document}

\maketitle


\newpage\null\thispagestyle{empty}\newpage


%%%%%%%%%%%%%%%%%%%%%%%%%%%%%%%%%%%%%%%%%%%%%%%%%%%%%%%%%%%%%%%%%%%%%%%%%%%%%%%%%%%%%%%%
\thispagestyle{empty}


\noindent\includegraphics[width=\textwidth]{cabecera}\vspace{1cm}

\noindent D. Jesus Enrique Sierra García, profesor del departamento de Ingeniería, área de Lenguajes y Sistemas Informáticos.

\noindent Expone:

\noindent Que el alumno D. \nombre, con DNI 71306054M, ha realizado el Trabajo final de Grado en Ingeniería Informática titulado Detección de palets mediante LIDAR. 

\noindent Y que dicho trabajo ha sido realizado por el alumno bajo la dirección del que suscribe, en virtud de lo cual se autoriza su presentación y defensa.

\begin{center} %\large
En Burgos, {\large \today}
\end{center}

\vfill\vfill\vfill

% Author and supervisor
\begin{minipage}{0.45\textwidth}
\begin{flushleft} %\large
Vº. Bº. del Tutor:\\[2cm]
D. nombre tutor
\end{flushleft}
\end{minipage}
\hfill
\begin{minipage}{0.45\textwidth}
\begin{flushleft} %\large
Vº. Bº. del co-tutor:\\[2cm]
D. nombre co-tutor
\end{flushleft}
\end{minipage}
\hfill

\vfill

% para casos con solo un tutor comentar lo anterior
% y descomentar lo siguiente
%Vº. Bº. del Tutor:\\[2cm]
%D. nombre tutor


\newpage\null\thispagestyle{empty}\newpage




\frontmatter

% Abstract en castellano
\renewcommand*\abstractname{Resumen}
\begin{abstract}
Cada vez es más la tecnología que se implementa en la industria de cara a mejorar sus procesos haciendolos más eficientes. \\Dentro del transporte de mercancias, ya sea a nivel interno como a nivel externo de una industria, se han visto importantes mejoras en tecnología en los ultimos años, como la popularización de los AGV's(Automated Guided Vehicle). \\ \\
Los AGV's son vehiculos automatizados que se usan para transportar cargas, por lo general paletizadas,  a traves de las zonas de las fábricas o almacenes.
Normalmente, estos vehículos tienen rutas preprogramadas para recoger la mercancía, pero ¿qué pasa si la mercanciía que iban a recoger no se encuentra exactamente donde tenía que estar?\\
Una desviación en el ángulo de un palet, una separación de unos metros respecto de donde debía estar, pueden hacer que el AGV no sea capaz de recoger ese pallet, con la consecuencia del paro en esa línea de transporte y la necesidad de una intervención humana para volver a retomar el funcionamiento.\\
Aquí es donde entra en juego la visión artificial, dotar a los AGV de la capacidad de detectar la mercancía y su posición, para así volverlos más precisos y eficientes y no depender de la correcta colocación de la mercancía. \\ \\
Existen multitud de opcionas para el reconocimiento de los pallets en cuanto a sensores se refiere, pero se ha optado por utilizar un sensor de tipo LIDAR por su capacidad de funcionamiento independientemente de las condiciones de luminosidad, funcionando incluso en total oscuridad.\\ Otros sensores como cámaras RGB no son capaces de proporcionarnos este rango de operacion, teniendo el inconveniende de perder mucha precisión en el momento que las condiciones de luminosidad no son las adecuadas y llegando a ser prácticamente inservibles en total oscuridad.\\
En este proyecto se investiga como utilizar un sensor láser LIDAR para el reconocimiento de palets en el entorno simulado de un almacen.\\
\end{abstract}

\renewcommand*\abstractname{Descriptores}
\begin{abstract}
LIDAR, detección de palets, AGV, láser, nube de puntos \ldots
\end{abstract}

\clearpage

% Abstract en inglés
\renewcommand*\abstractname{Abstract}
\begin{abstract}
Technology nowadays is more and more implemented in the industry in order to enhance it's process and make them more efficient. \\
Refering to logistics, whether it is at internal level or at external level to the industry itself, there has been mayor improvements in last years technology, e.g. the popularization of AGV's (Automatic Guided Vehicles).\\
AGV's are automated vehicles used for cargo transportation, tipically palletized, through the different zones of a warehouse or a factory.\\
Tipically, AGV's follow preprogrammed routes to pick cargo up, but, what if the cargo that was going to be picked is not exactly where it is supposed to be?\\
A little angle deviation, a small separation from the original position of the cargo, can make the AGV not able to pick the cargo up, slowing down or even stopping that transportation route and needing human intervention to restore normal operation.\\
This is where artificial vision systems come into play, giving AGV's the capability to to detect cargo and its position, to make them more efficient and precise, so they don't depend on the exact colocation of the cargo. \\
There are plenty of options to recognise pallets if we talk about sensor types, but in this particular case it has been chosen to use a LIDAR sensor due to its capacity to work independently of the lighting conditions, \\
beeing able to work even at complete darkness. Other sensor types like RGB cameras are not able to provide this vast range of operation, with the disadvantage of loosing a lot of presition when the lighting conditions are not optimal and even beeing almost useless at total darkness. \\
In this project it is investigated how to use a laser LIDAR sensor to recognise pallets at a simulated warehouse enviroment.\\
\end{abstract}

\renewcommand*\abstractname{Keywords}
\begin{abstract}
LIDAR, pallet detection, AGV, laser, point cloud \ldots
\end{abstract}

\clearpage

% Indices
\tableofcontents

\clearpage

\listoffigures

\clearpage

\listoftables
\clearpage

\mainmatter
\capitulo{1}{Introducción}

\section{Introducción}
Cada vez es más la tecnología que se implementa en la industria de cara a mejorar sus procesos haciendolos más eficientes. \\Dentro del transporte de mercancias, ya sea a nivel interno como a nivel externo de una industria, se han visto importantes mejoras en tecnología en los ultimos años, como la popularización de los AGV's(Automated Guided Vehicle). \\ \\
Los AGV's son vehiculos automatizados que se usan para transportar cargas, por lo general paletizadas,  a traves de las zonas de las fábricas o almacenes.
Normalmente, estos vehículos tienen rutas preprogramadas para recoger la mercancía, pero ¿qué pasa si la mercanciía que iban a recoger no se encuentra exactamente donde tenía que estar?\\s \\
Una desviación en el ángulo de un palet, una separación de unos metros respecto de donde debía estar, pueden hacer que el AGV no sea capaz de recoger ese pallet, con la consecuencia del paro en esa línea de transporte y la necesidad de una intervención humana para volver a retomar el funcionamiento.\\
Aquí es donde entra en juego la visión artificial, dotar a los AGV de la capacidad de detectar la mercancía y su posición, para así volverlos más precisos y eficientes y no depender de la correcta colocación de la mercancía. \\ \\
Existen multitud de opcionas para el reconocimiento de los pallets en cuanto a sensores se refiere, pero se ha optado por utilizar un sensor de tipo LIDAR por su capacidad de funcionamiento independientemente de las condiciones de luminosidad, funcionando incluso en total oscuridad.\\ Otros sensores como cámaras RGB no son capaces de proporcionarnos este rango de operacion, teniendo el inconveniende de perder mucha precisión en el momento que las condiciones de luminosidad no son las adecuadas y llegando a ser prácticamente inservibles en total oscuridad.\\
En este proyecto se investiga como utilizar un sensor láser LIDAR para el reconocimiento de palets en el entorno simulado de un almacen.\\

\section{Estructura de la memoria}
La memoria se divide según la siguiente estructura:
\begin{itemize}
\tightlist
\item
  \textbf{Introducción:} Descripción  resumida del problema a resolver y solución propuesta al mismo. 
Estructura de la memoria.
\item
  \textbf{Objetivos del proyecto:} Objetivos que pretende alcanzar el proyecto
\item
  \textbf{Conceptos teóricos:} Explicación de los conceptos teóricos no triviales tratados en el proyecto para la comprensión de la solución.
\item
  \textbf{Técnicas y herramientas:} Listado de técnicas metodológicas y
  herramientas utilizadas para gestión y desarrollo del proyecto.
\item
  \textbf{Aspectos relevantes del desarrollo:} Explicación de los aspectos
  destacables que se han dado durante la realización del proyecto.
\item
  \textbf{Trabajos relacionados:} Estado del arte en la investigación de los AGV's autónomos y reconocimiento de materiales paletizados.
\item
  \textbf{Conclusiones y líneas de trabajo futuras:} Conclusiones
  a las que se ha llegado tras la realización del proyecto y potenciales mejoras y lineas de desarrollo futuras.
\end{itemize}
\newpage
Junto a la memoria se proporcionan los siguientes anexos:

\begin{itemize}
\tightlist
\item
  \textbf{Plan del proyecto software:} Planificación temporal y estudio
  de viabilidad del proyecto.
\item
  \textbf{Especificación de requisitos del software:} Se describe la
  fase de análisis; los objetivos generales, el catálogo de requisitos
  del sistema y la especificación de requisitos funcionales y no
  funcionales.
\item
  \textbf{Especificación de diseño:} Se detalla la fase de diseño; el
  ámbito del software, el diseño de datos, el diseño procedimental y el
  diseño arquitectónico.
\item
  \textbf{Manual del programador:} Recoge los aspectos más relevantes
  relacionados con el código fuente (estructura, compilación,
  instalación, ejecución, pruebas, etc.).
\item
  \textbf{Manual de usuario:} Guía de usuario para el correcto manejo de
  la aplicación.
\end{itemize}



\capitulo{2}{Objetivos del proyecto}

Este apartado explica cuales van a ser los objetivos que persigue el proyecto. Tratando de diferenciar los objetivos generales de partida y los objetivos técnicos que aparecen a lo largo del desarrollo.

\section{Objetivos generales}
Los objetivos generales del proyecto se resumen en:
\begin{itemize}
	\item Establecer la conexion con el dispositivo láser para ser capaz de recibir tramas de información desde un ordenador.
	\item Interpretar los datos de lectura recibidos y efectuar el adecuado tratamiento de los mismos.
	\item Concluir si en los datos de lectura existe o no un palet y su posición en el caso de existir.
\end{itemize}

\section{Objetivos técnicos}
Los objetivos tecnicos del proyecto son los siguientes:
\begin{itemize}
	\item Desarrollar un algoritmo capaz de tratar los datos recibidos en forma de puntos y procesarlos en tiempo real.
	\item Desarrollar una aplicación en python para la recepción y tratamiento de los datos.
	\item Establecer un umbral para poder decidir si en los datos recibidos se visualiza un palet y conocer la situación del laser respecto del palet en téminos de distancia.
	\item Utilizar la arquitectura Modelo-Vista-Controlador MVC (Model-View-Presenter MVP)
	\item Usar durante el desarrollo del proyecto una herramienta de control de versiones, en este caso GitHub
	\item Hacer uso de la metodología ágil Scrum para el desarrollo del proyecto.
\end{itemize}

\section{Objetivos personales}
\begin{itemize}
\item Aprender sobre las distintas aproximaciones al problema de la detección de palets.
\item Hacer uso de los conocimientos aprendidos a lo largo de la carrera.
\item Ser capaz de utilizar herramientas y conocimientos ajenos a la carrera mediante el autoaprendizaje.



\capitulo{3}{Conceptos teóricos}

La parte del algoritmo más compleja reside en el propio algoritmo de detección de palets y en el tratamiento de los datos recibidos desde el láser.
El flujo del programa se compone de los siguientes pasos:
\begin{itemize}
	\item Preprocesado de los datos: \\
	\begin{enumerate}
		\item Primero se establece una conexión con el LIDAR, mediante TCP.
		\item Posteriormente se tratan los datos para su interpretación en el algotimo detector.
	\end{enumerate}
\imagen{diagramaflujo1}{Diagrama superficial del flujo de la aplicación}
	\item Algoritmo detector de palet: \\
	\begin{enumerate}
		\item Se selecciona un área de interes donde se conoce previamente que puede estar el palet.
		\item Sobre la nube de puntos ya acotada se realiza un clustering con el algoritmo KMEANS.
		\item Sobre los datos resultantes de la ejecución del algoritmo de agrupamiento, se realiza un conteo de los puntos que conforman cada cluster.
		\item Se comprueba que el número de puntos de cada cluster tenga un tamaño igual al del resto de clusters aplicando una tolerancia ajustada por el programador.
		\item Se hallan la distancia y ángulo medio a cada uno de los clusters.
		\item En función de esos datos se calcula la separación real entre los clusters.
		\item Se contrasta dichas separaciones entre sí mismas y con el dato real de distancia entre las patas del palet, de nuevo con cierta tolerancia.
		\item Se comunica la detección positiva o negativa del palet en función de las comprobaciones anteriores.


	\end{enumerate}
En el siguiente diagrama de flujo se representa el algoritmo anteriormente descrito. \emph{F} corresponde a la distancia fija que separa las patas de un palet, tomada de las medidas reales.
\imagen{Diagramaflujo2}{Diagrama del algoritmo de detección}
	\end{itemize}

\newpage
\section{Preprocesado de los datos}

Las tramas que se reciben del láser tienen en su interior información que no nos interesa para el fin del proyecto, como es el estado del mismo, datos de control etc. con lo cual se criban los datos descartando información innecesaria.\\


Posteriormente, ya que el LIDAR utilizado trabaja en datos hexadecimales, para poder procesarlos, se debe realizar una traducción de los mismos a formato decimal.\\
La traducción se realiza de la siguiente manera:\\
	\subsection{Traducción}

	Una vez extraidos los datos de cada apartado de la cadena devuelta por el láser, estos deben ser traducidos en base a los 			siguientes pasos:
	\begin{enumerate}
		\item Se extrae uno por uno el dato en hexadecimal de los caracteres.
		\item Si este dato está comprendido entre $30_{h}$ y  $39_{h}$, se le resta  $30_{h}$. Mientras que si está entre  				$41_{h}$ y  $46_{h}$, se le resta $37_{h}$.
		\item Se convierte cada dato hexadecimal a binario.
		\item Se agrupan todos los datos en binario y se traduce la agrupación a decimal para expresar el dato en milímetros.
	\end{enumerate}
	\imagen{traduccion}{Traducción de los datos hexadecimal-decimal}
	
	Después, por cada dato traducido, se asigna a cada uno un ángulo correspondiente para expresar así la lectura en coordenadas polares.\\
	Posteriormente, se traducen los datos de coordenadas polares a coordenadas cartesianas mediante la siguiente formula: \\

	x = r * cos(\(\theta\))\\
	y = r * sin(\(\theta\))\\

\imagen{graficopuntos}{Gráfico en el que se muestra una lectura real del entorno}

	Ahora ya se dispone de una lista de distancia y ángulo, adecuada para poder ser representada en una gráfica y tratada en el algoritmo detector.\\


\section{Algoritmo de detección}

	\subsection{Clustering}
	El \emph{clustering} o \emph{algoritmo de agrupamiento} es un proceso por el cual se agrupan una serie de vectores en función de algun criterio \cite{wiki:clustering} .\\ 
Los criterios generalmente corresponden a cercanía o similitud entre los datos de los vectores.\\
En este caso particular se utiliza la función de distancia euclídea para el agrupamiento de los puntos de la nube, formando así grupos de puntos en función de su cercanía en el espacio.\\

		\subsection{K-Means}
		\emph{K-Means} o \emph{K-Medias} es un algoritmo de clustering que particiona \textit{n} datos en \textit{k} grupos.\\
Cada punto es agrupado en el conjunto cuyo valor medio es más próximo al del punto.
		\imagen{kmeans}{Algoritmo K-Means}
		Para comenzar, el algoritmo asigna \textit{k} centros aleatorios y realiza las agrupaciones en función de la distancia de los puntos a esos centros.\\
Después se actualizan los nuevos centros como el centro de cada cluster que se ha calculado.\\

		El algorimo es de carácter iterativo, ejecutándose varias veces hasta que la solución converge (Las soluciones de cada iteración no cambian) \cite{wiki:kmeans} .\\ \\

Se ejecuta el algoritmo K-means con k=3, puesto que la característica por la que se está intentando identificar el palet es por sus tres patas frontales, las cuales el láser va a observar de manera frontal, al ir este montado en la parte de las palas del AGV. \\

\subsection{Comprobaciones del algoritmo sobre el resultado de clustering}
Una vez agrupados los puntos que corresponden a cada pata,se comprueba que el número de cada cluster tiene un tamaño similar al resto (por motivos de ruido en las lecturas del láser, se aplica un margen de error especificado en el código).Concretamente se compararn el número de puntos que componen a cada cluster.\\ \\
Posteriormente, se calcula la distancia media a todos los puntos de cada cluster desde el emisor (en este caso el láser) así como el ángulo medio de los puntos de cada cluster.\\
Con esto conseguimos representar cada cluster por un punto (ángulo y distancia).\\
Ahora, mediante el teorema del coseno, y con los datos de distancia y ángulo de cada punto representande de los cluster, se calcula la distancia que separa a estos tres puntos.\\

\imagen{tcoseno}{Regla del coseno}

 Estos datos se contrastan con las medidas reales del tipo de palet que se esté utilizando. También se contrastan entre ellos para comprobar que ambas distancias son iguales, de nuevo con una tolerancia por los posibles errores de lectura.\\

\imagen{europaletmedidas}{Medidas de un palet de tipo europeo}


Una vez realizadas todas estas comprobaciones, en caso de haberlas superado, se puede concluir la existencia de un palet.


		









\section{Material necesario}

Se ha usado el siguiente hardware para la realización del proyecto:
\begin{itemize}
	\item Sensor Láser: Encargado de sondear el entorno y devolver los datos obtenidos. El láser que se va a utilizar es de tipo LIDAR (Laser Imaging Detector And Ranging).
Este tipo de láser emite haces de luz infrarroja, para despues recoger la reflexión de dichos haces de vuelta, y en función del tiempo entre la emisión y la recepción calcula la distancia.
En este proyecto, se ha utilizado el láser Hokuyo Safety Laser Scanner (UAM-05LP-T301), capaz de distinguir tres zonas independientes de detección en función de la distancia a la que se encuentren los objetos en el entorno.
\imagen{hokuyo}{Sensor láser Hokuyo utilziado en el proyecto}
	\item Cable ethernet: A través de este cable se envian y reciben datos del láser al ordenador. Se usará un UAM-NET, un cable Ethrernet de 3 metros de longitud desarrollado por Hokuyo, misma empresa desarrolladora del láser empleado, lo que hace que resulte idóneo para evitar problemas de incompatibilidad y asegurar así el correcto funcionamiento del sistema.
	\item Ordenador: Es la parte central del proyecto. Con él, se ejecuta el sistema software encargado de procesar los datos que recibe de la unidad láser, y mediante una serie de algortimos, concluir la detección o no de un palet en el área visionada.
\end{itemize}

\section{Tratamiento de los datos del láser}
El software desarrollado en este proyecto es el encargado de recibir los datos del láser, para su posterior procesado. Los datos se reciben en forma de tramas y se les debe aplicar un proceso de tratamiento para separar datos no relevantes de las tramas. Posteriormente se deben traducir estos datos de coordenadas polares a coordenadas cartesianas, y finalmente, ejecutar los algoritmos que determinan la posible detección de un palet.

\subsection{Uso del láser}
Para poder establecer la comunicación con el láser y ordenarle la captura de datos, se utilizan comandos. Aunque existen multitud de comandos disponibles, para nuestro objetivo nos bastara con utilizar los comandos de tipo AR, los cuales ordenan al laser devolver los datos de lectura.
Existen 6 comandos de este tipo.
\begin{enumerate}
	\item AR00: Medición única en la que devuelve las distancias.
	\item AR01: Medición única en la que devuelve las distancias e intensidades.
	\item AR02: Medición contínua en la que devuelve las distancias.
	\item AR03: Detiene la continuidad del comando AR02.
	\item AR04: Medición contínua en la que devuelve las distancias en intensidades.
	\item AR05: Detiene la continuidad del comando AR04.
\end{enumerate}

\subsection{Funcionamiento a nivel interno del láser}
				Este láser en concreto, escanea un ángulo de 270º mediante un cabezal rotativo que gira 2000rpm emisor de ondas infrarrojas.\cite{hokuyo:data_specification} 
\\ El láser lanza 1081 haces  a lo largo de los 270º de visión de los que posee, mientras la unidad receptora recoge la reflexión de cada uno de esos haces en los objetos del entorno. Así, calcula el tiempo que ha tardado cada emisión en retornar y, sabiendo la velocidad a la que se propagan los haces, computa la distancia a la que se encuetra cada punto.\\
En este apartado cabe destacar los problemas que surgieron derivados del amplio rango de captura del láser, puesto que su funcionalidad principal es la de escaner de seguridad y no de medición. Al tener un rango tan grande, en las capturas efectuadas existia mucho ruido del entorno que dificultaba la correcta identificación del palet, con lo cual se optó por limitar el rango con el que se trabajaba a nivel de código para así focalizar la atención del algoritmo en la parte frontal del láser.\\
			


\subsection{Tránsito de mensajes}
Una vez determinado el comando que se quiere utilizar, se debe crear una estructura de mensaje para enviarselo al láser con el siguiente formato:
\begin{itemize}
	\item STX: Tipo caracter, en bytes, que marca el comienzo del mensaje. Normalmente es un '2'.
 	\item Command size: Tamaño del mensaje que se va a enviar. Formato hexadecimal. En este proyecto, los comandos van a ocupar siempre el mismo espacio, 14 caracteres, con lo que esta parte del mensaje siempre será '000E'.
 	\item Header: Tipo de comando que se le manda al laser, como se ha mencionado antes, en este proyecto solo se usa el tipo AR.
 	\item Subheader: Especificación del comando que se va a utilizar dentro de la familia de comandos escogida en la cabecera.
 	\item CRC:  Comprobacion de redundacia Cíclica.Código que se añade para asegurar que el mensaje no se ha corrompido en el envío del mismo. 
 	\item EXT: Tipo caracter, en bytes, que marca el final del mensaje. Normalmente es un '3'.
 \end{itemize}

En los mensajes devueltos por el láser al pedirle información de una lectura mediante un comando, se encuentra información como el tiempo empleado para la lectura, estado de los puertos etc que no tiene utilidad en este proyecto. Es por esto que debe ser llevado a cabo un proceso de traducción para extraer la información que nos interesa de las tramas que nos devuelve el láser de las lecturas.






\capitulo{4}{Técnicas y herramientas}

Esta parte de la memoria tiene como objetivo presentar las técnicas metodológicas y las herramientas de desarrollo que se han utilizado para llevar a cabo el proyecto. 


\section{Herramientas para el desarrollo de la memoria}
Para el desarrollo de la memoria del proyecto se ha usado LaTex como procesador de textos, principalmente por su libertad y simplicidad a la hora de manipular los distintos parametros del documento. Además, desde LaTex se puede exportar directamente el archivo a PDF de una manera muy sencilla.
Como desventaja de este programa, cabe destacar el aprendizaje necesario al no haberlo utilizado antes a lo largo del grado en contraposición a la facilidad de haber utilizado un procesador de textos en el que tuviera más experiencia como por ejemplo Microsoft Word u Open Office.

\section{Herramientas para el desarrollo y pruebas del código}

\subsection{Realterm}
Este programa se ha utilizado para la conexión con el dispositivo láser. Su funcionamiento se resume en un programa para tramsmitir mensajes TCP a través del puerto serie. Así, a través de esta conexión, establecemos una via de comunicación con el láser de tipo Cliente-Servidor.

\section{Metodología}
\subsection{Scrum}
Scrum es una metodología denominada \emph{ágil}.
Se realiza un desarrollo de manera incremental, dividiendo el proyecto en subproblemas más pequeños.
Cada subtarea pertenece a un \emph{sprint} tras el cual se realiza una revisión de las tareas realizadas.
Se ha elegindo esta metodología para llevar a cabo el proyecto dado a su facilidad para ser flexible ante cambios y nuevos requisitos del desarrollo.

\section{Patrones de diseño}
\subsection{Model View Controller (MVC)}
MVC consta de tres componentes a nivel de software:
\begin{itemize}
\item Modelo: Representación de los datos.
\item Vista: Parte que 'muestra' la información al usuario y se comunica con él. COmunmente una interfaz de usuario.
\item Controlador: Es la parte encargada de realziar las peticiones al modelo y gestionar los eventos que el usuario genera.

Es un patrón de diseño adecuado puesto que separa en capas claramente distinguibles y funcionalmente similares el software desarrollado, facilitando el mantenimiento y desarrollo de código.
\imagen{mvc}{Patrón MVC}

\section{Gestión del repositorio}
Para la gestión del proyecto y la creación y actualización de un repositorio online, se han barajado dos herramientas: \emph{Github} y \emph{BitBucket}.
Finalmente se ha optado por usar Github debido a  mayor habituación de uso a lo largo de la carrera en diversas asignaturas.
\subsection{GitHub}
Github es una plataforma online para el control de versiones, basada en el sistema Git. Integra funcionalidades como documentación, revision de código, gestion de bugs, de tareas etc.
La herramienta es gratuita para proyectos de código abierto, sin embargo, para crear un repositorio privado es necesaria una suscripción.

\section{Gestión de las tareas}
Existen multitud de herramientas y programas que se pueden utilizar para implementar scrum, como por ejemplo Jira, Trello, QuickScrum etc.
Por razones de comodidad se ha optado por Trello, dado que es un software que ya se ha usado en la carrera, es intuitivo y su uso es fácil de aprender.

\section{IDE (Entorno de desarrollo Integrado)}
\subsection{Python}


\section{Documentación}

\subsection{LaTex}
\capitulo{5}{Aspectos relevantes del desarrollo del proyecto}

\section{Conocimientos aplicados en el proyecto}

Son muchos los conocimientos que han sido necesarios de una u otra forma para llevar a cabo este proyecto. Muchos de esos conocimientos se han adquirido a lo largo de las asignaturas de este grado, mientras que otros han sido obtenidos mediante el autoaprendizaje haciendo uso de distintos tipos de recursos \textit{Manuales de programación, tutoriales, documentación online...)}\\
	
\subsection{Conocimientos aprendidos dentro del grado}
	\subsubsection{Programación en Python}
		Este lenguaje ha sido clave para desarrollar todo el código del proyecto, un lenguaje que se ha ido aprendiendo a lo largo del grado en diversas asignaturas como Sistemas 					Inteligentes, Algoritmia, Nuevas Tecnologías y Empresa y Gestión de la Información.\\
		Las nociones y conocimientos sobre este lenguaje que se han recibido han servido para poder realizar la parte de programación de este proyecto, así como para facilitar el 					aprendizaje de nuevas librerías y características necesarias para la realización del código.\\
	\subsubsection{Redes}
		Para manejar la conexión necesaria con el láser, se han empleado conocimientos sobre redes, concretamente sobre conexiones TCP y sockets. \\ Esto ha sido posible gracias a lo 				aprendido en asignaturas como Redes o Sistemas Distribuidos.\\
	\subsubsection{Gestión de proyectos}
		La gestión de proyectos es la parte encargada en el proyecto de planear las tareas a realizar para llegar al objetivo final, adecuando los tiempos, costes tanto humanos como 				materiales y todo lo necesario. \\ Como ya se ha expuesto previamente, se ha utilizado una metodología ágil, SCRUM, la cual además de otros conocimientos necesarios para la 				gestión ha sido aprendida en la asignatura Gestión de Proyectos. También se ha aprendido en esta asignatura el concepto de los tableros KanBan %\citep{wiki:Tableros Kanban}
		, un método similar a la metodología SCRUM.\\
Se han seguido desarrollos incrementales mediante sprints de una duración comprendida entre un mes y tres meses en las partes más complejas del proyecto, retrasándose cuando se arrastraban tareas de sprints anteriores por fallos en el código, entre otros contratiempos. Para seguir el desarrollo de los sprints se utilizó la aplicación \href{https://trello.com/}{Trello}.\\
	\subsubsection{Hardware}
		El aspecto a nivel de hardware del proyecto comprende  la alimentación electrica del propio láser y la conexión de datos con el mismo.\\
		Para la alimentación se buscó una fuente de alimentacion de corriente continua, 24v y por lo menos 1A como se especifica en los manuales del equipo láser.
		En cuanto a la conexión de datos con el láser, fue necesario comprender que función desempeñaba cada cable (\textit{Polo positivo, polo negativo, tierra, reset...})\\
		Fue en la asignatura Mantenimiento de Equipos Informáticos donde se adquirieron los conocimientos necesarios para operar con el equipo láser y su fuente de alimentación.\\

	
	\subsection{Conocimientos aprendidos fuera del grado}


		\subsubsection{Manejo del equipo láser}
			Durante el desarrollo del proyecto ha sido necesario comprender cómo funciona el láser a nivel interno, qué distintos comandos usa y cómo funciona el protocolo de 						comunicación para poder operar con él y aprovechar sus funcionalidades\\

		\subsubsection{Investigación del estado del arte}
			Con el fin de conocer el nivel de desarrollo de este problema a nivel global,  se ha aprendido a la búsqueda de artículos científicos en diversas plataformas como son \href{https://www.researchgate.net/}{ResearchGate}, \href{https://scholar.google.es/}{Google Scholar} y \href {https://ieeexplore.ieee.org/Xplore/home.jsp}{IEE Explore} entre otros.\\
Se ha adquirido la capacidad de filtrado a la hora de decidir qué artículos representaban información relevante para el proyecto y cuales no, así como la capacidad de sintetizar el contenido de los artículos para conocer otras alternativas y aproximaciones al problema de reconocimiento de palets.\\

\imagen{ejemploscholar}{Ejemplo de búsqueda en Google Scholar}


			
			

\section{Desarrollo del algoritmo de detección}
		Buena parte del esfuerzo del proyecto se dedicó al desarrollo y programación del algoritmo para la detección de los palets.\\
		Se empezó por documentarse sobre otras soluciones al problema de detección y una clasificación de las mismas. Se encontraron multitud de artículos sobre el problema con diversas aproximaciones. \\ Posteriormente, basándose en el equipo láser disponible y su forma de capturar los datos en una nube de puntos 3D, se estudió la mejor colocación posible de cara a su potencial incorporación a un AGV. Se llegó a la conclusión de que la mejor perspectiva para observar el palet era a ras de suelo, montado sobre las propias palas del AGV de manera que el palet quede en el mismo plano que el láser. \\
\imagen{montajelaser}{Punto óptimo de instalación del sensor láser en un AGV}

		Se optó por desarrollar el algoritmo en el lenguaje Python en contraposición a Java en base a su mejor adecuación para aplicaciones en tiempo real, por su amplio uso y soporte y por ser un lenguaje ampliamente usado durante el grado y así poder aplicar los conocimientos aprendidos.\\

		El algoritmo de desarrollo mediante prueba y error, añadiendo pequeños incrementos de funcionalidad a lo largo del desarrollo, como podían ser añadir nuevas comprobaciones a la hora de la detección o hacer las que ya estaban implementadas más robustas.\\

		Han sido varias las alternativas al algoritmo que se han estudiado, pero finalmente, optando por la simplicidad y dejando abierto el desarrollo de mejoras en la detección, se optó por un camino en el que la detección se basa en el reconocimiento del frontal del palet mediante las patas del mismo, y la distancia que las separa.\\
En la siguiente imagen se pueden observar desde un plano cenital las tres patas del palet, situadas aproximadamente a un metro del láser.

\imagen{tramapalet}{Trama de datos en la que se puede identificar visualmente las patas del palet detectado}



\section{Desarrollo del apartado gráfico}

		Inicialmente, se comenzó con la implementación de una interfaz gráfica en \href{https://docs.python.org/2/library/tkinter.html}{TKinter}
, pero la necesidad de aprender e investigar sobre esta librería de Python y varios problemas de funcionamiento causaron el cambio a otras librerias más sencillas.\\
Posteriormente, con el fin de poder observar de una manera sencilla pero a la vez suficientemente detallada los datos en tiempo real que transmite el láser, se ha utilizó en lugar de TKinter, las animaciones de la biblioteca \href{https://matplotlib.org/}{matplotlib}.Tras varios problemas de implementación que causaban que la gráfica no se refrescara, se consiguió solucionar el funcionamiento.\\ De esta manera el usuario puede observar en tiempo real las tramas que el láser está procesando, junto con la confirmación de palet en la trama actual. 




\capitulo{6}{Trabajos relacionados}

En este apartado del proyecto se ha realizado una labor de investigación en el campo de los AGV's y la detección de objetos, profundizando en la detección de palets.
Existen múltiples aproximaciones a este problema, en las que se usa una variedad de técnicas y sensores para intentar resolver la problemática.

\section{Artículos científicos}

\subsection{Detection, localisation and tracking of pallets using machine
learning techniques and 2D range data}
En 2018, los autores Ihab S. Mohamed, Alessio Capitanelli, Fulvio Mastrogiovanni, Stefano Rovetta y Renato Zaccaria publicaron este estudio en el cual presentan una arquitectura capaz de dotar a un AGV, mediante un sensor láser de tipo LIDAR similar al utilizado en el proyecto, de mantener un conteo de palets detectados, asi como su posición.
La arquitectura se divide en dos componentes:
\begin{itemize}
\item En primer lugar, un detector de palets que utiliza una red neuronal convolucional basado en zonas (Region-based convolutional neural network), en cascada con un clasificador.
\item En segundo lugar, se utiliza un fltro Kalman para rastrear los palets ya detectados.

\subsection{Robust Pallet Detection for Automated Logistics Operations}
En este documento publicado en 2016 por Robert Varga y Sergiu Nedevschi, se estudian las alternativas ya existentes para la visión artificial de los AVG's y se proponen soluciones para mejorarlas.
Se utilizan para la detección imagenes captadas por una cámara estereoscópica y imagenes en escala de grises. Estas últimas proporcionan la información de la localización del palet en 2D mientras que las primeras se usan para conocer la posición 3D y la ortientación del palet respecto de la cámara.
Emplean un algoritmo compuesto de detección de bordes y líneas de los palets, para después generar posibles candidatos a ser considerados un palet, y probarlos contra las características ya conocidas. Por último, se usa un clasificador para confirmar si se trata o no de una detección positiva.

\subsection{Focus based Feature Extraction for Pallets Recognition}
En este estudio realizado por Rita Cucchiara, Massimo Piccardi y Andrea Prati se utiliza una cámara estandar para extraer características de un palet sobre la imagen que se esta capturando, concretamente se buscan las patas laterales y la pata central del palet mediante el procesado de líneas y bordes, después de haber definido una región de ineterés (ROI). Posteriormente mediante un arbol de decisión se confirma si es o no un palet lo que se esta viendo.

\subsection{Visualization of Pallets}
Roger Bostelman, Tsai Hong y Tommy Chang publicaron este estudio en 2006 en el que utilizan dos sensores de tipo LIDAR, uno en la base del AGV y otro en las 'palas'.
Este estudio se especializa en la detección de palets dentro de un camión, de modo que usan las paredas laterales y la pared del fondo del camión para conocer la posición del palet.
El algoritmo utilizado realiza los siguientes pasos:
\begin{enumerate}
\item Se convierten los datos obtenidos en plano cartesiano X Y
\item El sistema estima la localización de las cuatro esquinas del camión. (x1, y1), (x2, y2), (x3, y3), (x4, y4).
\item Se aplica la transformada de Hough a los puntos.
\item Encuentra las líneas que delimitan el camión a ambos lados y el fondo.
\item Si alguna de las paredes del camión concuerda con las predicciones, la distancia desde el centro del sensor a esta pared se sobreescribe.
\item Se calcula la distancia hacia cada una de las tres paredes de los puntos que están dentro de los límites predichos.
\item Se agrupan los puntos cuya distancia a la pared izquierda es menor a la pared contraria como el palet izquierdo y viceversa.
\item Se calcula la distancia mínima del sensor a cada uno de los palets encontrados.
\end{enumerate}

 \subsection{Pallet recognition and localization using an RGB-D camera}
Autores: Junhao Xiao, Huimin Lu, Lilian Zhang y Jianhua Zhang.
En este estudio se untiliza una cámara RGB, concretamente la Kinect 2.0 de Microsoft.
En primer lugar el algoritmo realiza un segmentado de las imágenes y filtrado, para después compararlas con muestras de palets.

\subsection{Automatic visual guidance of a forklift engaging a pallet}
En este estudio, Michael Seelingera, y John-David Yoderb usan dos cámaras estándar para capturar imágenes en escala de grises.
No obstante, también hacen uso de marcas visuales tanto en el propio AGV como en los palets, y aún siendo un sistema igualmente válido, posee la desventaja de tener que equipar todos los palets del almacen o fábrica con dichas marcas. Puesto que en el ámbito del transporte de mercancias los palets 'viajan' mucho entre la industria, es uan tarea demasiado laboriosa mantener todos los palets que pasan por un almacen o fábrica con las marcas necesarias.

\subsection{Pallet Pose Estimation with LIDAR and Vision for Autonomous Forklifts}
Los autores de este estudio (N. Bellomo, E. Marcuzzi, L. Baglivo, M. Pertile, E. Bertolazzi y M. De Cecco) proponen un método de reconocimiento de palets que extrae los datos de un sensor de tipo LIDAR, y utiliza un algortimo genético de optimización para encontrar la mejor coincidencia entre la imagen que se esta recibiendo, y un modelo existente de un palet conocido.

\subsection{Feature-to-Feature based Laser Scan Matching in Polar Coordinates with Application to Pallet Recognition}
Este artículo publicado en 2011 por Zhendong Hea, Yaonan Wanga y Hong Yuc hace uso de un sensor de tipo LIDAR para recoger la información del entorno.
Después, su algoritmo emplea segmentación y clustering sobre los puntos recibidos por el LIDAR, detección de las esquinas del palet y finalmente lo compara con un modelo de palet conocido para concluir si se trata de un palet.

\subsection{Docking to Pallets with Feedback from a Sheet-of-Light Range Camera
}
Los autores J. Nygirds, T. Hogstrom y A. Wernersson presentan en este estudio un algoritmomas que para el reconocimiento de palets, para la correcta inserción de las palas del AGV una vez es conocida ya la existencia del palet. Hace uso de dos tipos de cámaras: una cámara estándar, y otra de tipo 'sheet of light' (Se trata de una cámara que recoge la reflexion de un haz de luz proyectado para así conseguir una imagen en tres dimensiones.
Se compone de los siguientes pasos:
\begin{enumerate}

\item Se define una región de interés (ROI) en base a la predicción del palet en la imagen.
\item Haciendo uso de la transformada de Radon se detecta la orientación del palet y la distancia al mismo.
\item Buscando los intervalos en los que los valores de intensidad del láser son mayores, se determina cual es la pata central del palet.
\item Se transforman las distancias de los puntos obtenidas a coordenadas para el AGV.
\end{enumerate}


\subsection{Vision-based autonomous load handling for automated guided vehicles}
En este estudio de 2014, Robert Varga y Sergiu Nedevschi presentan un método para detectar la existencia de un pallet y estimar su posición y orientación en el espacio.
Se utilizan dos cámaras posicionadas en el AGV, una entre las palas del AGV que se mueve solidariamente con ellas, y otra en una posicion fija.
También se dispone de una fuente de luz compuesta por varios leds para garantizar las condiciones de iluminación necesarias.
Mediante las dos cámaras se generan imágenes en 3D, que posteriormente se tratan con filtros y una ventana deslizante para elegir los candidatos a pallets de la imagen obtenida.


\subsection{Real-time Pallet Localization with 3D Camera Technology for Forklifts in Logistic Environments}
En este estudio, se utiliza una cámara Kinect v2 de Mircrosoft montada en las palas del AGV. Mediante un algoritmo de región creciente (Region growing algorithm) y segmentación de las nubes de puntos que se reciben como entrada, se buscan particularidades que identifican los pallets, concretamente los bloques de las patas, y la superficie que los une. Cuando se detecta un candidato a pallet, se comprueban las medidas que la cámara esta viendo con las medidas reales de un pallet europeo con un pequeño margen de error y así verificar el pallet.





\capitulo{7}{Conclusiones y Líneas de trabajo futuras}

En este apartado se expresan las conclusiones del proyecto y las posibles mejoras y lineas de trabajo futuras para continuar con el proyecto.

\section{Conclusiones}

Después de la realización del proyecto se ha llegado a las siguientes conclusiones:

\begin{itemize}
\item El objetivo del proyecto se ha cumplido con ciertas salvedades.\\
	Se ha conseguido realizar un programa con un algoritmo capaz de detectar la existencia de palets utilizando el sensor láser en tiempo real, mostrando el proceso al usuario y calculando satisfactoriamente la distancia y el angulo a las patas del palet detectado. No obstante las condiciones en las que se puede reconocer un palet no han sido suficientemente probadas en diferentes entorno y condiciones.
La robustez de la detección es suficiente pero mejorable.

\item Usar un sistema láser que nunca antes había utilizado ha supuesto un reto de adaptación y aprendizaje para poder llegar a manejar y comprender su funcionamiento y la transmisión de datos.

\item Se ha aprendido a buscar soluciones a los contratiempos y poder continuar con el desarrollo, algo que ha sido muy común a lo largo de todo el proceso tanto en el entorno de programación como con el algoritmo en sí o las herramientas utilizadas.

\item Se han empleado gran parte de los conocimientos aprendidos a lo largo del grado asi como otros conocimientos que se han obtenido durante la realización del proyecto.

\item Se han descubierto nuevas herramientas y familiarizado con su uso las cuales resultaran útiles para desarrollos futuros.

\item Se ha conocido como llevar a cabo un proyecto de investigación sobre un tema prácticamente desconocido como era la visión artificial. Comprendiendo la estructura y pasos del mismo.

\item Se ha aprendido a realizar una búsqueda sobre el estado del arte de un problema o tema, mediante búsquedas bibliográficas y el descubrimiento de bibliotecas online donde encontrar artículos científicos y más.

\item Se han sintetizado muchos conocimientos del grado que se aprenden de manera aislada. Es decir, se ha obtenido una visión global de como todos esos conocimientos son necesarios y colaboran de manera intrínseca a la hora de llevar a cabo un proyecto real, aunque haya sido de carácter académico.

Recapitulando, el desarrollo de este proyecto aunque en ocasiones frustrante y muy demandante ha sido por lo general un proceso de aprendizaje provechoso en el que se ha descubierto más en detalle los problemas que van surgiendo y como llevar a cabo su investigación y desarrollo.

\end{itemize}

\section{Líneas de trabajo futuras}

\subsection{Mejoras en la detección de palets}
La detección de palets puede ser mejorada para ser más robusta y precisa, realizando un tratamiento de los datos más exhaustivo.\\
Además, se planteó durante el desarrollo la incorporación de una red neuronal en contraposición al algoritmo empleado, haciendo uso de un algoritmo de aprendizaje automatizado, que con una base de datos de palets fuera entrenado para después funcionar como clasificador.

\subsection{Desarrollo de una interfaz gráfica}
Aunque el destino de la aplicación vaya a ser su instalación en un AGV, queda abierto a desarrollos futuros la incorporación de una interfaz gráfica que permita al operario comprender y utilizar de una manera más sencilla e intuitiva el programa. 

\subsection{Seguridad del código}
En futuras iteraciones se puede mejorar la robustez del código frente a excepciones, valores fuera de rango etc.

\subsection{Prueba en un entorno real}
Por limitaciones en el tiempo y lugar donde se ha realizado el proyecto, no se ha podido probar el funcionamiento del laser y el programa en un entorno real de una fábrica y almacen, sino que se ha simulado el encuentro del laser con un palet en un laboratorio, lejos de presentar las condiciones a las que se enfrentará en un entorno real.







%\bibliographystyle{plain}
%\bibliography{bibliografia}


\end{document}
