\documentclass[11pt]{}
\capitulo{1}{Introducción}
\begin{I}
\section{Introducción}
Cada vez es más la tecnología que se implementa en la industria de cara a mejorar sus procesos haciendolos más eficientes. El ámbito del transporte de mercancias, ya sea a nivel interno como a nivel externo de una industria, ha visto importantes mejoras en tecnología en los ultimos años, como la popularización de los AGV's(Automated Guided Vehicle).
Estos, se usan para mover cargas a través de las zonas de una fábrica o cargar y descargar camiones entre otras.
Normalmente, estos vehículos tienen rutas programadas para recoger la mercancía, y no se desvían de estas, pero ¿qué pasa si lo que iban a recoger no se encuentra exactamente donde tenía que estar?
Aquí entra en juego la visión artificial, dotar a los AGV de la capacidad de detectar la mercancía y su posición, para así volverlos más precisos y eficientes.
En este proyecto se investiga como utilizar un sensor láser LIDAR para el reconocimiento de palets en el entorno simulado de un almacen.
\end{I}
